\documentclass[russian, 14pt, a4paper]{extreport}
\usepackage{extsizes} 
\usepackage{cmap} 
\usepackage[T2A]{fontenc}
\usepackage[utf8x]{inputenc}
\usepackage[russianb]{babel}
\usepackage{graphicx}
\usepackage{titlesec}
\usepackage{amssymb, amsmath, amsfonts, amsthm}
\usepackage{longtable}
%\usepackage[nottoc, notlot, notlof]{tocbibind}
\linespread{1.3}
\usepackage{vmargin}
\usepackage{scrbase}
\usepackage{hyperref}
\usepackage{listings}
\usepackage{siunitx}
\usepackage{extsizes}
\usepackage{cmap}
\usepackage{scrbase}
\usepackage[titles]{tocloft}
\usepackage{titlesec}
\usepackage[nottoc, notlot, notlof]{tocbibind}
\usepackage{indentfirst}
\sloppy

\binoppenalty=10000
\relpenalty=10000

\titlespacing\chapter{0pt}{12pt plus 4pt minus 2pt}{0pt plus 2pt minus
  2pt} \titlespacing\section{0pt}{12pt plus 4pt minus 2pt}{0pt plus
  2pt minus 2pt}

\titlelabel{\thetitle.\quad}

\titleformat{\chapter}[hang]{\filcenter\bfseries\large}{\thechapter}{1em}{\bfseries}{}
\titleformat{\section}[hang]{\filcenter\bfseries\normalsize}{\thesection}{1em}{\bfseries}{}

\renewcaptionname{russian}{\contentsname}{Содержание}
\renewcaptionname{russian}{\bibname}{Список использованной литературы}

\setmarginsrb{2cm}{1.5cm}{1cm}{1.5cm}{0pt}{0mm}{0pt}{13mm}

\renewcommand{\cftchapleader}{\cftdotfill{\cftdotsep}}
\renewcommand{\cftchapfont}{\normalfont}
\renewcommand{\cftchappagefont}{\normalfont}
\renewcommand{\cftchapaftersnum}{.}

\begin{document}

\lstset{
  basicstyle=\ttfamily,
  columns=fullflexible,
  keepspaces=true,
  fontadjust=true,
  basewidth=0.5mm
}

\begin{titlepage}
  \begin{center}
    Федеральное государственное образовательное бюджетное учреждение\\
    высшего профессионального образования\\
    "<Сибирский государственный университет телекоммуникаций и\\
    информатики">
  \end{center}
  \vfill \vfill
  \begin{flushright}
    Кафедра вычислительных систем
  \end{flushright}
  \vfill \vfill \vfill
  \begin{center}
    \begin{bfseries}
      {\large Пояснительная записка к курсовой работе}\\
      По дисциплине: Моделирование\\
      На тему: Моделирование вычислительных систем с различными способами извлечения задач из очереди\\
    \end{bfseries}
  \end{center}
  \vfill \vfill
  \begin{flushright}
    Выполнили:\\
    студенты гр. МГ-155\\
    Чирихин~К.С.\\
    Варежникова~М.А. \vfill
    Проверил:\\
    ассистент кафедры ВС\\
    Скиба~А.И.
  \end{flushright}
  \vfill \vfill \vfill \vfill \vfill \centerline{Новосибирск, 2016}
\end{titlepage}
\setcounter{page}{2}
\tableofcontents
\chapter{Введение}
Под имитационным моделированием понимается построение имитационной модели системы и её экспериментальное исследование на ЭВМ \cite{book:rad}. При этом модель должна описывать процессы так, как если бы они происходили в действительности. С такой моделью можно провести как одно испытание, так и некоторое их множество. При этом результаты будут определяться случайным характером процессов. После проведения экспериментов выполняется обработка статистических данных и делаются выводы.

К имитационному моделированию прибегают, когда:
\begin{itemize}
\item долго или невозможно проводить эксперименты над реальным объектом;
\item невозможно построить аналитическую модель;
\item необходимо сымитировать поведение системы во времени.
\end{itemize}

В данной работе используется дискретно-событийный подход к моделированию. Он заключается в абстрагировании от непрерывной природы событий и перехода к рассмотрению только основных событий моделируемой системы. Применительно к ЭВМ, это могут быть такие события, как поступление задачи в очередь, отказ ядра и т.д.

\textbf{Целью} данной работы является разработка и программная реализация модели многоядерной ЭВМ с очередью задач. Предполагается, что ЭВМ работает в мультипрограммном режиме. Необходимо проверить гипотезу о равенстве среднего времени ожидания задачи в очереди при различных дисциплинах очереди: 
\begin{itemize}
\item FIFO (First Input First Output) --- первый пришел - первый вышел;
\item SL (Slowest Last) --- самая "<тяжёлая"> задача извлекается из очереди последней. Критерием сравнения задач является время их выполнения. Если задачу, требующую минимальное количество временных ресурсов, невозможно поместить в вычислительную систему (ВС) на выполнение, возможны два варианта:а) не помещать в ВС ни одной новой задачи, пока не удастся поместить данную задачу, и б) поместить в ВС подходящую задачу с минимальным временем выполнения. Реализовать оба варианта;
\item SF (Slowest First) --- самая "<тяжёлая"> задача извлекается из очереди первой. В двух вариантах, по аналогии с предыдущим пунктом.

\end{itemize}
\chapter{Описание модели ВС}
Будем считать, что задаче требуется определённое количество оперативной памяти и ядер. В соответствии с поставленной задачей, организация работы с памятью, отказы ядер и др. не являются существенными и моделироваться не будут. Если в ВС поступает новая задача, возможны два варианта: 1) в ВС свободно достаточно ресурсов для начала её выполнения; 2) в ВС не достаточно памяти или ядер, и задаче придётся провести некоторое время в очереди. Чтобы обобщить эти два случая, поступим следующим образом: вновь поступившая задача всегда помещается в очередь. Сразу же после этого происходит попытка извлечения задачи из очереди. Также необходимо пробовать извлечь задачу из очереди при завершении выполнения какой-либо задачи. Получаем три события: прибытие задачи в ВС, постановка задачи на выполнение и завершение выполнения задачи. Данные о времени нахождения задачи в очереди можно снимать непосредственно в самой очереди, т.к. через неё пройдут абсолютно все задачи, в том числе и те, время пребывания которых в ней будет равно 0.

Граф событий представлен на рисунке \ref{fig:graph}. На этом рисунке событие 1 --- поступление задачи в ВС, событие 2 --- постановка задачи на выполнение, событие 3 --- завершение выполнения задачи, условие I --- в ВС достаточно свободной памяти и ядер, в очереди нет более приоритетных задач, условие II --- очередь не пуста, в ВС достаточно свободной памяти и ядер, в очереди нет более приоритетных задач.
\begin{figure}
  \centering
  \includegraphics{graph}
  \caption{Граф событий}
  \label{fig:graph}
\end{figure}

Так как будем работать в рамках дискретно-событийного подхода, в начале работы программы можно сформировать календарь событий. Каждое событие характеризуется тремя атрибутами: время, в которое оно должно произойти, действие, которое нужно совершить, а также имя для отладки. В этот календарь заранее можно добавить только события "<Поступление задачи в ВС">, т.к. события "<Завершение выполнения задачи"> и "<Постановка задачи на выполнение"> добавить невозможно ввиду неизвестности времени начала выполнения задачи. События этих типов должны быть добавлены в календарь в процессе работы модели, при постановке задачи в ВС на выполнение. Каждая задача также характеризуется тремя атрибутами: требуемое количество памяти, ядер и время её выполнения.

Первые два атрибута задачи являются дискретными случайными величинами. Распределение вероятностей по ядрам будем задавать при помощи ряда распределения. Диапазон возможных требований памяти широк, и указать все возможные значения данного параметра вручную сильно затруднительно. Поэтому было решено использовать следующий подход. Пользователь задаёт квантили функции распределения ячеек памяти с постоянным шагом, т.е. уровней \(\alpha_{i} - \alpha_{i - 1} = C,\;i = 1, 2, \ldots, n,\;C = 1/n\). Таким образом, количество интервалов равно \(n\), все интервалы равновероятны. Генерируем дискретную случайную величину, принимающую значения \(0, 1, \ldots, n-1\) с равными вероятностями, это и будет номер интервала. Все значения из каждого интервала дискретны с шагом \(\Delta\), например, \(\Delta = 16\), и равновероятны. Конкретное значение из интервала выбирается аналогично, путём "<бросания"> случайной величины. Время поступления задачи в очередь и время её выполнения --- непрерывные случайные величины \(\xi_{1}\) и \(\xi_{2}\), распределённые по показательному закону с параметрами \(\tau_{1}\) и \(\tau_{2}\), генерируемые методом обратной функции:
\[\eta_{i} = F_{\xi_{i}}(x) = 1 - e^{\tau_{i}x} \Rightarrow x = -\dfrac{1}{\tau_{i}}\ln \eta_{i},\]
где \(\eta_{i}\) --- случайная величина, равномерно распределённая на отрезке \([0;1]\).

\chapter{Программная реализация}
Программа была написана с использованием языка программирования C++. Основные классы:
\begin{itemize}
\item \verb:Queue: --- реализация очереди задач в ВС. Любая задача, поступающая в систему, обязательно попадает в объект этого класса. Здесь же накапливается статистика о времени пребывания задач в очереди;
\item \verb:Sl_queue, Sf_queue, Sl_queue_opt, Sf_queue_opt: --- классы, производные от \verb:Queue:. Переопределяют методы добавления (извлечения) задач в (из) очереди;
\item \verb:CS: --- класс, реализующий модель вычислительной системы. Содержит ячейки памяти и ядра, а также методы для их выделения и освобождения;
\item \verb:Event: --- базовый класс для всех типов задач.
\end{itemize}

Также есть функции для работы с календарём событий: \verb:schedule, simulate:.

Параметры модели задаются в конфигурационном файле, для чтения такого файла используется библиотека \verb:libconfig: \cite{web:lc}. Пример такого файла:
\lstinputlisting[breaklines=true]{main.cfg}

Память задаётся в килобайтах. 
\chapter{Экспериментальное исследование и обработка результатов}
Будем проверять гипотезу \(H_0\) о том, что среднее время пребывания задач в очереди одинаково, для всех пар типов очередей. Для этого воспользуемся двусторонним \(t\)-критерием Стьюдента. Статистическая обработка экспериментальных данных была выполнена в программе R, результаты теста Стьюдента в которой представлены в виде \(p\)-значений. \(p\)-value --- вероятность ошибки первого рода, или, иначе говоря, это наименьшее значение уровня значимости, для которого вычисленная проверочная статистика ведёт к отклонению нулевой гипотезы \cite{web:ds}.

В результате запуска программы с приведённым ранее конфигурационным фалом и несколькими другими аналогичными файлами, отличающимися только типом очереди, были получены данные о времени пребывания задачи в очереди, их графики приведены на рисунке \ref{fig:queue_plot}.

\begin{figure}[t]
  \centering
  \includegraphics[width=0.8\textwidth]{./lab1/plot}
\caption{Зависимость времени пребывания задачи в очереди от номера задачи}
\label{fig:queue_plot}
\end{figure}

Как видно из рисунка \ref{fig:queue_plot}, при дисциплине SF чаще, чем при других дисциплинах, задача пребывает в очереди достаточно долго (для разных очередей различный масштаб). Для методов, в которых выбирается наиболее подходящая задача, чаще всего время ожидания минимально, хотя, иногда, оно может быть очень большим. Среднее время пребывания задачи в очереди приведено в таблице \ref{tab:mean}.
\begin{table}[t]
  \centering
  \begin{tabular}[h]{|l|l|}
    \hline Дисциплина & Время, c. \\ \hline
    FIFO & 890.0333 \\ \hline
    SL & 185.5955 \\ \hline
    SF & 2186.998 \\ \hline
    SL с выбором подходящей задачи & 97.08909 \\ \hline
    SF с выбором подходящей задачи & 82.34653 \\ \hline
  \end{tabular}
  \caption{Среднее время пребывания задачи в очереди для различных типов очередей}
  \label{tab:mean}
\end{table}
Очевидно, что средние не совпадают для всех дисциплин, кроме SL с подбором и SF с подбором. Воспользуемся \(t\)- критерием Стьюдента, код на языке R будет выглядеть следующим образом:
\begin{verbatim}
t.test(queue.sf.opt, queue.sl.opt, alternative = "two.sided")
\end{verbatim}
Результат:
\begin{verbatim}

	Welch Two Sample t-test

data:  queue.sf.opt and queue.sl.opt
t = -1.6511, df = 15718, p-value = 0.09873
alternative hypothesis: true difference in means is not equal to 0
95 percent confidence interval:
 -32.244105   2.759003
sample estimates:
mean of x mean of y 
 82.34653  97.08909 
\end{verbatim}

При уровне значимости \(\alpha = 0.05\) гипотеза \(H_{0}\) не должна быть отклонена, следовательно, различий в средних не выявлено.

Проверим гипотезу о равенстве средних и для дисциплин SL и SL с подбором. Получим:
\begin{verbatim}

	Welch Two Sample t-test

data:  queue.sl and queue.sl.opt
t = 7.7141, df = 14562, p-value = 1.297e-14
alternative hypothesis: true difference in means is not equal to 0
95 percent confidence interval:
  66.01723 110.99558
sample estimates:
mean of x mean of y 
185.59549  97.08909 

\end{verbatim}

Это означает, что, если гипотеза \(H_{0}\) верна, то вероятность того, что разница в статистике для обеих выборок будет больше или равной полученной, составляет \num{1.297e-14}, следовательно, при уровне значимости \(\alpha = 0.05\) мы вынуждены отклонить эту гипотезу.
\chapter{Заключение}
В процессе выполнения курсового проекта была разработана и реализована дискретно-событийная модель вычислительной системы. Путём исследования результатов имитационного моделирования статистическими методами было установлено, что дисциплина извлечения задач из очереди влияет на среднее время пребывания задачи в очереди, за исключением дисциплин с подбором наилучшей задачи. В последнем случае гипотеза о равенстве средних не противоречит действительности.
\begin{thebibliography}{00}
\bibitem{book:rad} Родионов А.С. Учебное пособие по имитационному моделированию.
\bibitem{web:lc} [Электронный ресурс] libconfig --- C/C++ Configuration File Library URL: \url{http://www.hyperrealm.com/libconfig/} 
\bibitem{web:ds} [Электронный ресурс] p-значение. URL: \url{http://datascientist.one/p-value/}  
\end{thebibliography}
\end{document}

%%% Local Variables:
%%% mode: latex
%%% TeX-master: t
%%% End:
